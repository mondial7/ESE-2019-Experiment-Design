% This is samplepaper.tex, a sample chapter demonstrating the
% LLNCS macro package for Springer Computer Science proceedings;
% Version 2.20 of 2017/10/04
%
\documentclass[runningheads]{llncs}
%
\usepackage[utf8]{inputenc}
\newtheorem{hypothesis}{Hypothesis}
\usepackage{graphicx}
\usepackage{color}
% Used for displaying a sample figure. If possible, figure files should
% be included in EPS format.
%
% If you use the hyperref package, please uncomment the following line
% to display URLs in blue roman font according to Springer's eBook style:
% \renewcommand\UrlFont{\color{blue}\rmfamily}

\begin{document}
%
\title{ESE 2019 - Experiment Design}
%
%\titlerunning{Abbreviated paper title}
% If the paper title is too long for the running head, you can set
% an abbreviated paper title here
%
\author{Marco Mondini}
%
\authorrunning{M. Mondini}
% First names are abbreviated in the running head.
% If there are more than two authors, 'et al.' is used.
%
\institute{Universidad Politécnica de Madrid\\
\email{marco.mondini@alumnos.upm.es}}
%
\maketitle              % typeset the header of the contribution
%
\begin{abstract}
Software development teams deal on a daily base with decisions that can affect the project at different levels. Tools and practices to support the decision-making process have been addressed by the literature over the past decades. The Information Radiator is one of these tools adopted by practitioners, often in very customised shapes. Considering the context of software start-ups, where the development team works closely with the others, with this experiment I want to evaluate the effects on the decision-making process while using an Information Radiator meant for the development team and populated with non-technical information. Whether such kind of information (i) motivate the team and (ii) drive teams re-actions.

\keywords{Information Radiator \and Startups \and Development Team \and Metrics}
\end{abstract}
%
%-=
%
\section{Experiment formalisation}

\begin{hypothesis}
Team-, market- and finance-related information presented on an information radiator affects the motivation and the decisions of the development team as product related information do.
\end{hypothesis}

\begin{table}
\caption{Experiment formalisation}\label{tab1}
\begin{center}
\begin{tabular}{p{6cm}|p{6cm}}
\hline
\hline
\textbf{Factor} & Kind of information item\\
\hline
\textbf{Treatments} & Product (technical), Team, Market, Finance information\\
\hline
\textbf{Response Variables} & Team motivation, Number of decisions/actions based on the provided pieces of information\\
\hline
\textbf{Response Variables Measurements} & Motivation assessed with post-experiment self-evaluation\newline Desisions assessed through observations\\
\hline
\hline
\end{tabular}
\end{center}
\end{table}


\begin{center}
\textbf{RQ1.} \textit{Does the kind of the information item affects the motivation\\ and decision-making of the development team?}
\end{center}

In order to answer this question, we can perform an experiment with four different groups, separating information items related to the product, also considered technical, from team-, market- and finance-related items.
Two possible design applicable to this experiment can be within-subjects and between-subjects.

\subsection{Between-subjects Design}

While one of the general goals of this experiment is assessing the influence of non-technical metrics on the Development team, it would be interesting to start to scope down such differences and analyse more than technical vs non-technical. To this direction, a between-subjects design (Table \ref{tab2}) would allow us to properly control the environment and save resources to perform the experiment, by managing a single session. On the other side, assuming we would have a fixed population available, splitting it into four groups would decrease the number of subjects per group. However, because of the possibility to avoid practice-biases, by not allowing the groups to interacts with each other until the end of the session, this design seams to properly fit the experiment. In fact, the practice-biases are crucial in this study given the response variables related to \textit{motivation} and \textit{decision-making}.

\begin{table}
\caption{Design Table Between-subjects}\label{tab2}
\begin{center}
\begin{tabular}{p{3.7cm}|p{2cm}|p{2cm}|p{2cm}|p{2cm}}
\hline
\hline
\textbf{1 Factor\newline4 Treatments} & \textbf{Group 1} & \textbf{Group 2} & \textbf{Group 3} & \textbf{Group 4}\\
\hline
\textbf{Session 1} & Product (technical) & Team & Market & Finance\\
\hline
\hline
\end{tabular}
\end{center}
\end{table}

\subsection{Within-subjects Design}

As alternative to the between-subjects design we could perform the study on the same population repeatedly. In such within-subjects design we would have a single group and multiple sessions as can be seen in Table \ref{tab3}. This would increase the risk of introducing confounding variables. In particular, the design would be transparent to the subjects and we are measuring social aspects of the group. In this way we would be introducing threats to the validity, for example by the order of the treatments. Moreover, another negative note of this design would be the time necessary to perform it. The most appropriate population for the experiment is represented by the software development team of a company. This would require the company to invest the team in multiple sessions. To avoid this limitation we could redesign the experiment to perform each session with a different group, e.g. a different company team, going back to a between-subjects experiments with multiple sessions. Finally, even this option would lead to confounders.

\begin{table}
\caption{Design Table Within-subjects}\label{tab3}
\begin{center}
\begin{tabular}{p{6cm}|p{6cm}}
\hline
\hline
\textbf{1 Factor\newline4 Treatments} & \textbf{Group 1}\\
\hline
\textbf{Session 1} & Product (technical)\\
\hline
\textbf{Session 2} & Team\\
\hline
\textbf{Session 3} & Market\\
\hline
\textbf{Session 4} & Finance\\
\hline
\end{tabular}
\end{center}
\end{table}

In conclusion, the chosen design is a single session between-subjects experiment (Table \ref{tab2}) with a particular attention to the randomised assignment of the population to each group, in order to preserve internal validity.

\section{Threats to validity}


\subsection{Conclusion validity}

The response measures of this experiment lack of simple well defined metrics to be collected. Thus, I will deal with metrics to measure motivation and decision-making that will lack of reliability. In order to face this threat, the decision-making measure has been simplified to a count of the decisions taken during the experiment sessions. This does not completely solve the threat since we are introducing the assumption that the judgement of the researcher won't affect the identification of all the actual decisions taken. On the other side from what concerns the motivation, the lack of a theoretical framework to use in order to collect such data drove me to a design that would involve only the use of likert scale questionnaire to evaluate the degree of motivation of the participants during the session. However, we are limiting the collection to instant motivation that could actually differ much from the long term motivation. For such long term motivation, more longitudinal studies should probably be conducted. Finally, regarding the treatment implementation, the researcher will be responsible for clearly identify the decision taken among the different categories of information items studied. Biases of the researcher could then lead to miss some group decisions. To prevent this, the presence of a second observer could contain such biases effect.

\subsection{Internal validity}

Given the goals and the response variables of the experiment it will be critical to avoid any interference between groups of ideas during the groups discussions. This would affect the internal validity of the experiment, since a decision of a group might be related to decisions taken from other groups. In such scenario, it would be impossible to associate the decisions to the type of information item.

... need for more observers for each group, so more resources ...

... influenced effects on the response variables given by experience of participants ...

... influence of environment where the session is performed ...

\subsection{Construct validity}

In this experiments each group will be provided with a paper artefact showing a series of metrics/information items. After a short introduction the researcher will let the group discussion flow until saturation; considering a maximum time limit for the session, decided based on the number of participants per each group. In the introduction, the researcher would need to take care to not reveal expected outcomes of the experiment. Moreover, the participants should not be aware before or during the discussion about the way in which will be observed and evaluated at the end of the session.

Paper artefact ... why this way of operationalize?

\subsection{External validity}

Regarding the generalizibility of the results, this experiment will be affected mainly by the single session and the subjects involved. In details, the population might be a class of higher degree computer science students or a small company. In the latter, the experiment would be designed as a workshop. In order to generalize the results the same experiment could be conducted multiple times with different companies in different maturity levels. Thus, the results would be more representative of a general population.



%
% ---- Bibliography ----
%
% BibTeX users should specify bibliography style 'splncs04'.
% References will then be sorted and formatted in the correct style.
%
% \bibliographystyle{splncs04}
% \bibliography{mybibliography}
%
% \begin{thebibliography}{8}
% \bibitem{ref_article1}
% Author, F.: Article title. Journal \textbf{2}(5), 99--110 (2016)
% \end{thebibliography}
\end{document}
